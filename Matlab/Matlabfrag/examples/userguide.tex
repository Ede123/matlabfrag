\documentclass[a4paper,11pt]{article}

\usepackage[T1]{fontenc}\usepackage{lmodern}\usepackage[sc]{mathpazo}\linespread{1.1}% Palatino font
\usepackage[colorlinks,urlcolor=black,linkcolor=black]{hyperref}
\usepackage{upquote,fancyvrb,subfig,microtype,sistyle,refstyle}
\usepackage[process=auto,crop=preview]{pstool}

\newcommand\ytick[1]{\ensuremath{\mathcal R(e^{#1 i})}}

\newcommand\matlabfrag{\href{http://www.mathworks.com/matlabcentral/fileexchange/21286}{{\tt matlabfrag}}}
\newcommand\psfragname{\href{http://www.ctan.org/tex-archive/help/Catalogue/entries/psfrag.html}{{\tt psfrag}}}
\newcommand\pstoolname{\href{http://www.ctan.org/tex-archive/help/Catalogue/entries/pstool.html}{{\tt pstool}}}
\newcommand\matlab{{\sc Matlab}}
\newcommand\tex{{\sc .tex}}
\newcommand\eps{{\sc .eps}}
\def\'#1'{\texttt{\textquotesingle #1\textquotesingle}}

\title{\matlabfrag\ user guide}
\author{Zebb Prime}

\begin{document}
  \maketitle
  \tableofcontents
  \section{Introduction}%
    \matlabfrag\ is a function which exports a \matlab\ figure to \eps\ and \tex\ files for use in LaTeX
    with \psfragname. It is inspired by LaPrint, but is intended to be more WYSIWYG, by respecting figure
    handles better. The main reasons to use \matlabfrag\ are the same as those for using \psfragname: figure
    font matching that of the document, and the ability to have complex mathematical expressions typeset
    by LaTeX.
 
  \section{Background}%
    I wrote \matlabfrag\ after becoming frustrated with the default LaPrint behaviour, including it
    putting the line to insert the graphic in the output \tex\ file: \verb|\includegraphics{FileName.eps}|
    
    Whilst these problems could be addressed using one of the many options in LaPrint, I decided to
    take a stand against the many \matlab\ scripts and functions available which try to impose their
    own sense of style on my figures, and instead write a function which does everything possible to
    respect the figure handles. Everyone has their own sense of style, for better or worse, and I'm sure
    most users have their own little scripts set up to format their figures in their own way.
        
    The problem I have with the \verb|\includegraphics{FileName.eps}| line in the \tex\
    file is that I keep my figures in a {\tt graphics} subdirectory of my main document. This meant I had to
    manually edit the \tex\ file every time I exported the figure from \matlab.
    
  \section{Usage}%
    \subsection{Within \matlab}%
      Using \matlabfrag\ within \matlab\ is easy. Simply format the figure how you like it, then run
      \matlabfrag. The format for the \matlabfrag\ command is:
      \begin{verbatim}
matlabfrag(FileName,OPTIONS)
      \end{verbatim}
      where {\tt FileName} (required) is the file name for the output \tex\ and \eps\ files, and 
      {\tt OPTIONS} are key-value pairs for the optional options:
      \begin{itemize}
        \item[{\'handle'}] The handle for the figure to export. The default is \texttt{gcf} (`get 
		  current figure').
        \item[{\'epspad'}] The number of points to pad the output \eps\ file by. The default is \texttt{[0,0,0,0]}.
      \end{itemize}
      Example\nobreak
      \VerbatimInput[firstline=7,lastline=12]{ex01.m}
      See \Figref{ex01} for the output.
      \begin{figure}[ht]
        \centering
        \subfloat[graphics/ex01-1]{\fbox{\psfragfig{graphics/ex01-1}}}\quad
        \subfloat[graphics/ex01-2]{\fbox{\psfragfig{graphics/ex01-2}}}
        \caption{\matlabfrag\ options example showing the \texttt{epspad} option at work.}
        \figlabel{ex01}
      \end{figure}
      
      If you wish to show something different in \matlab\ and in the \tex\ document, then you
      can add it into the {\tt UserData} property of the text, with a {\tt matlabfrag:} prefix. This is
      especially useful if you have macros in LaTeX which you want to use, or if you want to use some
      commands not included in the \matlab\ LaTeX installation. In this example, the \verb|\LaTeX| macro
      typesets LaTeX as \LaTeX.
      \VerbatimInput[firstline=7,lastline=11]{ex02.m}
      See \Figref{ex02} for the output.
      \begin{figure}[ht]
        \centering
        \psfragfig{graphics/ex02}
        \caption{UserData example; X-label should say `Plays nice with \LaTeX'.}
        \figlabel{ex02}
      \end{figure}
      
    \subsection{Within LaTeX}%
      The \tex\ and \eps\ files can be included in a LaTeX document (pdfLaTeX is treated seperately below)
      by either loading the \psfragname\ package and including the \tex\ file before the \eps\ file:
      \begin{verbatim}
\documentclass{article}
\usepackage{graphicx,psfrag}
\begin{document}
  \include{FileName.tex}
  \includegraphics{FileName.eps}
\end{document}
      \end{verbatim}
      or my preferred method; loading the \pstoolname\ package (v1.2 onwards), and using the
      \verb|\psfragfig| macro:
      \begin{verbatim}
\documentclass{article}
\usepackage{pstool}
\begin{document}
  \psfragfig{FileName}
\end{document}
      \end{verbatim}
      Notice the lack of a file extension in the \verb|\psfragfig| macro. This is a requirement of \pstoolname.
      \pstoolname\ also loads the {\tt graphicx}, {\tt psfrag} and {\tt color} packages (as well as a few others),
      so it is not necessary to explicitly load these packages.
      
      For more information about \pstoolname\ or \psfragname\ please see their respective manuals.
      
    \subsection{Within pdfLaTeX}%
      There are several ways to include files with postscript commands (such as \psfragname) in pdfLaTeX such
      as {\tt pst-pdf}, {\tt auto-pst-pdf}, {\tt ps4pdf} and \pstoolname, by far the most useful of which is
      \pstoolname.  The document is set up the same way as the LaTeX example above:
      \begin{verbatim}
\documentclass{article}
\usepackage{pstool}
\begin{document}
  \psfragfig{FileName}
\end{document}
      \end{verbatim}
      The only difference being that \pstoolname\ runs some external processes to run the postscript commands,
      and as such it needs the \verb|-shell-escape| command when being called:\par
{\verb|pdflatex -shell-escape example.tex|}\par\noindent
      where {\tt example.tex} is the code above.
      
    \subsection{Within LyX}%
      If you are using LyX, you have to ensure \pstoolname\ is installed in your LaTeX distribution, then load
      the \pstoolname\ package by putting \verb|\usepackage{pstool}| in the preamble, enable
      \verb|-shell-escape| for when pdfLaTeX is executed, and include the figure using \verb|\psfragfig|
      as above in a \verb|ERT| box.
      
  \section{Creating figures in \matlab}%
    Hypocritically I am now going to suggest a few ways to manipulate figures in \matlab\ before exporting
    them using \matlabfrag.
    
    \subsection{Sizing}%
      I suggest that the first thing you do when creating a figure is setting it to the size that you want
      in the final document. This way the figure does not need to be resized at any stage, which prevents
      line sizes changing, text over-running borders, etc.  This can be done within \matlab\ using:
      \VerbatimInput[firstline=4,lastline=10]{ex03.m}
      See \Figref{ex03} for the output.
      \begin{figure}[ht]
        \centering
        \psfragfig{graphics/ex03}
        \caption{Resizing example; the figure is sized to \SI{8}{cm} by \SI{3}{cm}.}
        \figlabel{ex03}
      \end{figure}
      
    \subsection{{\tt Color}}%
      The {\tt color} property is a three element vector representing the RGB colour of the text.
      \matlabfrag\ will respect the colour set in \matlab, but may require that you load the {\tt color}
      package in LaTeX (\pstoolname\ implicitly loads the {\tt color} package).
      \VerbatimInput[firstline=7,lastline=11]{ex04.m}
      See \Figref{ex04} for the output.
      \begin{figure}[ht]
        \centering
        \psfragfig{graphics/ex04}
        \caption{{\tt Color} example; the text should be coloured.}
        \figlabel{ex04}
      \end{figure}
    
    \subsection{{\tt FontSize}}%
      All text in \matlab\ figures has a {\tt FontSize} property which \matlabfrag\ respects; the size
      specified in the figure is the size it will be in the LaTeX output document.
      \VerbatimInput[firstline=7,lastline=11]{ex05.m}
      See \Figref{ex05} for the output.
      \begin{figure}[ht]
        \centering
        \psfragfig{graphics/ex05}
        \caption{{\tt FontSize} example; the text should be different sizes.}
        \figlabel{ex05}
      \end{figure}

      
    \subsection{{\tt FontAngle}}%
      {\tt FontAngle} is a property shared by all text in a \matlab\ figure. \matlabfrag\ respects
      {\tt FontAngle} in that {\tt italic} and {\tt oblique} will be italic in the LaTeX output
      document.
      \VerbatimInput[firstline=7,lastline=11]{ex06.m}
      See \Figref{ex06} for the output.
      \begin{figure}[ht]
        \centering
        \psfragfig{graphics/ex06}
        \caption{{\tt FontAngle} example; Oblique and Italic font should both be italic here.}
        \figlabel{ex06}
      \end{figure}
      
    \subsection{{\tt FontWeight}}%
      Another text property from \matlab\ figures that \matlabfrag\ respects is {\tt FontWeight}, such
      that any text set to {\tt bold} will be made bold in the output LaTeX document. {\tt light},
      {\tt demi} and {\tt normal} have no effect on the font in the LaTeX output.
      \VerbatimInput[firstline=7,lastline=12]{ex07.m}
      See \Figref{ex07} for the output.
      \begin{figure}[ht]
        \centering
        \psfragfig{graphics/ex07}
        \caption{{\tt FontWeight} example; bold should be bold, demi and light do not really translate to
        LaTeX, so they should default to normal.}
        \figlabel{ex07}
      \end{figure}
            
    \subsection{{\tt FontName}}%
      {\tt FontName} is a property that \matlabfrag\ \emph{does not} respect, with one exception. One
      of the basic ideas behind using \matlabfrag\ and \psfragname\ is to match the font in figures to
      that of the text, so it does not make sense to use this property, except when it has been set to
      {\tt FixedWidth}, in which case the font in the output LaTeX document will be fixed-width (i.e.,
      \verb|\ttfamily|).
      \VerbatimInput[firstline=7,lastline=11]{ex08.m}
      See \Figref{ex08} for the output.
      \begin{figure}[ht]
        \centering
        \psfragfig{graphics/ex08}
        \caption{{\tt FontName} example; fixed-width should be in a typewriter font, while Comic Sans should
        be changed to the font of the document.}
        \figlabel{ex08}
      \end{figure}
      
    \subsection{{\tt Interpreter}}%
      \matlab\ has three (well, two really) text interpreters that is uses to render the text. These
      are {\tt tex}, {\tt latex} and {\tt none}. I generally don't recommend the use of the default
      {\tt tex} interpreter if rendering anything mathematical and using \matlabfrag, as while the {\tt tex}
      interpreter may render a mathematical expression fine, it may not work in LaTeX.
      
      The {\tt latex} interpreter is especially useful for rendering mathematical formula.
      \VerbatimInput[firstline=7,lastline=11]{ex09.m}
      See \Figref{ex09} for the output.
	    \begin{figure}[ht]
	      \centering
		  \psfragfig{graphics/ex09}
		  \caption{{\tt Interpreter} example; both equations should be the same, despite using a different
		    interpreter in \matlab.}
		  \figlabel{ex09}
	    \end{figure}
      
    \subsection{Multi-line text}%
      Multi-line text can be entered into \matlab\ using either a cell or a two-dimensional character
      matrix. If using a character matrix, \matlabfrag\ preserves all of the white spaces, so if you
      do not wish for this to occur, use cells instead (or check out the {\tt cellstr} function).
      
      \matlabfrag\ recreates multi-line text using a {\tt tabular} environment. If you also use the
      {\tt UserData} property, you will need to manually put the {\tt tabular} environment in, as
      \matlabfrag\ will ignore any {\tt UserData} that is not a string.
	    \VerbatimInput[firstline=7,lastline=13]{ex10.m}
	    See \Figref{ex10} for the output.
	    \begin{figure}[ht]
	      \centering
		    \psfragfig{graphics/ex10}
		    \caption{Multi-line example; three different ways to make multi-line text. 2D arrays, cells
		      and using {\tt tabular}.}
		    \figlabel{ex10}
	    \end{figure}
	  
	  \subsection{Custom tick labels}%
	    Custom tick labels can be added directly as strings for LaTeX to interpret. In this example,
	    the \verb|\ytick| macro has been defined in the preamble of this document as:
      \par{\verb|\newcommand\ytick[1]{\ensuremath{\mathcal R(e^{#1 i})}}|}\par\noindent
      The \matlab\ example code is:
      \VerbatimInput[firstline=7,lastline=12]{ex11.m}
      See \Figref{ex11} for the output.
      \begin{figure}[ht]
        \centering
        \psfragfig{graphics/ex11}
        \caption{Custom tick label example.}
        \figlabel{ex11}
      \end{figure}
	  
  \section{Frequently (occasionally) asked questions}%
    \subsection*{Why does the output have [`number'] in it?}
	
	  Either you are looking at the \eps\ output and that is how it is supposed to look,
	  or there was a problem processing the image with \psfragname. Check out the troubleshooting section
	  below.
	
	  \subsection*{Why doesn't the \tex\ file have the {\tt includegraphics} statement in it like
	  LaPrint does?}
	
	  Firstly, read the Background section above. With all the image files stored in a
	  {\tt graphics} subdirectory of the main document, I would have to manually open the \tex\ file and
	  insert the path into the that line. This was one of the reasons I wrote \matlabfrag\ in the first
	  place.
	
	  I have considered including an option to produce this output for those who would rather use this
	  behaviour than load \pstoolname\ but decided against it due to the directory problems. From v1.2 onwards
	  \pstoolname\ works fine in LaTeX so there really is no need to manually put the \verb|\includegraphics|
	  line in.
	
    \section{Troubleshooting}	
	  \begin{enumerate}
	    \item Check that all LaTeX statements in your figure are valid. If you are having problems, try a
	      simple example that is known to work.
	    \item If using pdfLaTeX and \pstoolname, check:
	    \begin{enumerate}
	      \item \verb|-shell-escape| option is set when calling pdfLaTeX.
	    \end{enumerate}
	    \item If using LaTeX and \pstoolname, check:
	    \begin{enumerate}
	      \item \pstoolname\ is at least at version 1.2. An alternative location to finding \pstoolname\ is
		    \url{http://github.com/wspr/pstool/}.
	    \end{enumerate}
	    \item If using LaTeX or pdfLaTeX with anything else:
	    \begin{enumerate}
	      \item Install \pstoolname !
	    \end{enumerate}
	    \item If using LyX:
	    \begin{enumerate}
	      \item Learn how to use LaTeX!
	    \end{enumerate}
	    \item Shut up and reboot!
	    \item Perhaps you have found a bug. If so, please email the smallest self-contained example that
	      reproduces the bug to me at\par\noindent{\tt zebb.prime+matlabfrag@gmail.com}
	  \end{enumerate}
  
\end{document}