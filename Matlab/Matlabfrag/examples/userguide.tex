\documentclass[a4paper,11pt]{article}

\usepackage[T1]{fontenc}\usepackage{lmodern}\usepackage[sc]{mathpazo}\linespread{1.1}% Palatino font
\usepackage[colorlinks,urlcolor=black,linkcolor=black]{hyperref}
\usepackage{upquote,fancyvrb,subfig,microtype,sistyle,refstyle,wasysym}
\usepackage[process=auto,crop=preview]{pstool}

\newcommand\ytick[1]{\ensuremath{\mathcal R(e^{#1 i})}}

\newcommand\matlabfrag{\texorpdfstring{\href{http://www.mathworks.com/matlabcentral/fileexchange/21286}{{\ttfamily matlabfrag}}}{matlabfrag}}
\newcommand\psfragname{\texorpdfstring{\href{http://www.ctan.org/tex-archive/help/Catalogue/entries/psfrag.html}{{\ttfamily psfrag}}}{psfrag}}
\newcommand\pstoolname{\texorpdfstring{\href{http://www.ctan.org/tex-archive/help/Catalogue/entries/pstool.html}{{\ttfamily pstool}}}{pstool}}
\newcommand\matlab{\texorpdfstring{{\sc Matlab}}{Matlab}}
\newcommand\tex{\texorpdfstring{{\sc .tex}}{.tex}}
\newcommand\eps{\texorpdfstring{{\sc .eps}}{.eps}}
\newcommand\pdf{\texorpdfstring{{\sc .pdf}}{.pdf}}
\def\'#1'{{\ttfamily \textquotesingle #1\textquotesingle}}

% Get rid of that stupid S thing in section references.
\newref{sec}{%
  name    = \RSsectxt,
  names   = \RSsecstxt,
  Name    = \RSSectxt,
  Names   = \RSSecstxt,
  refcmd  = {\ref{#1}},
  rngtxt  = \RSrngtxt,
  lsttxt  = \RSlsttxt}

\title{\matlabfrag\ user guide}
\author{Zebb Prime}

\begin{document}
  \maketitle
  \tableofcontents
  \section{Introduction}%
    \matlabfrag\ is a function which exports a \matlab\ figure to \eps\ and \tex\ files for use in LaTeX
    with \psfragname. It is inspired by LaPrint, but is intended to be more WYSIWYG, by respecting figure
    handles better. The main reasons to use \matlabfrag\ are the same as those for using \psfragname: figure
    font matching that of the document, and the ability to have complex mathematical expressions typeset
    by LaTeX.
 
  \section{Background}%
    I wrote \matlabfrag\ after becoming frustrated with the default LaPrint behaviour, including it
    putting the line to insert the graphic in the output \tex\ file: \verb|\includegraphics{FileName.eps}|
    
    Whilst these problems could be addressed using one of the many options in LaPrint, I decided to
    take a stand against the many \matlab\ scripts and functions available which try to impose their
    own sense of style on my figures, and instead write a function which does everything possible to
    respect the figure handles. Everyone has their own sense of style, for better or worse, and I'm sure
    most users have their own little scripts set up to format their figures in their own way.
        
    The problem I have with the \verb|\includegraphics{FileName.eps}| line in the \tex\
    file is that I keep my figures in a {\ttfamily graphics} subdirectory of my main document. This meant I had to
    manually edit the \tex\ file every time I exported the figure from \matlab.
    
  \section{Compared to LaPrint}%
    In this section I compare the output from {\ttfamily LaPrint} and \matlabfrag\ for an identical
    figure. I have chosen some examples that shows some of the weaknesses of {\ttfamily LaPrint}, so
    be sure to take these comparisons with a grain of salt.
    
    The code for the first comparison is given in\par
    {\verb|examples/comparison01.m|}\par\noindent
    and the results are shown in \Figref{comparison01}. Notice that the legend box for the LaPrint
    output has been shrunk, and is too small for the equation, and the y axis scale has disappeared.
    I am unwilling to blame {\ttfamily LaPrint} for the equation not rendering as it may by a bug
    in any stage of the fairly complicated conversion process (which has nothing to do with
    \matlabfrag). The output \tex\ and \eps\ files appear to have the text conversion set up
    properly.
    \begin{figure}[ht]
      \centering
      \subfloat[graphics/comparison01-matlabfrag]{\figlabel{comparison01-matlabfrag}%
      \fbox{\psfragfig{graphics/comparison01-matlabfrag}}}
      \,
      \subfloat[graphics/comparison01-laprint]{\fbox{\psfragfig{graphics/comparison01-laprint}}}
      \caption{\matlabfrag\ output versus {\ttfamily LaPrint} output for a simple graph. Notice the
        legend box for the {\ttfamily LaPrint} is incorrectly sized, and the y axis scale has
        disappeared. I am unsure why the equation is not rendering in the {\ttfamily LaPrint}
        version, and it is unlikely to be the fault of {\ttfamily LaPrint}.}
      \figlabel{comparison01}
    \end{figure}
    
    In this second comparison a scaled version of the \matlab\ {\ttfamily peaks} function is
    presented. The \matlab\ code for this example is given in:\par
    {\verb|examples/comparison02.m|}\par\noindent
    and the results are shown in \Figref{comparison02}. The axis labels are handled better by
    \matlabfrag\, and it also manages to reproduce the x and y axis scales, but I stress that they
    only work well for the default orientation of a {\ttfamily surf} plot. If you start rotating
    the image they are unlikely to come out correctly. This is discussed further in \Secref{adv-axis}.
    \begin{figure}[ht]
      \centering
      \subfloat[graphics/comparison02-matlabfrag]{\fbox{\psfragfig{graphics/comparison02-matlabfrag}}}
      \,
      \subfloat[graphics/comparison02-laprint]{\fbox{\psfragfig{graphics/comparison02-laprint}}}
      \caption{\matlabfrag\ output versus {\ttfamily LaPrint}.}
      \figlabel{comparison02}
    \end{figure}
    
  \section{Usage}%
    \subsection{Within \matlab}%
      Using \matlabfrag\ within \matlab\ is easy. Simply format the figure how you like it, then run
      \matlabfrag. The format for the \matlabfrag\ command is:
      \begin{verbatim}
matlabfrag(FileName,OPTIONS)
      \end{verbatim}
      where {\ttfamily FileName} (required) is the file name for the output \tex\ and \eps\ files, and 
      {\ttfamily OPTIONS} are key-value pairs for the optional options:
      \begin{itemize}
        \item[{\'handle'}] The handle for the figure to export. The default is {\ttfamily gcf} (`get 
		  current figure').
        \item[{\'epspad'}] The number of points to pad the output \eps\ file by. The default is {\ttfamily [0,0,0,0]}.
        \item[{\'renderer'}] The output renderer. The default is {\ttfamily painters}. The renderer is
          discussed in more detail in \Secref{renderers}.
        \item[{\'dpi'}] The output resolution (DPI or more appropriately, PPI) of the figure.
          For the {\ttfamily painters} renderer this makes little difference, but is important when
          using the {\ttfamily opengl} or {\ttfamily zbuffer} renderers. The default is 300. A
          discussion of renderers is given in \Secref{renderers}.
      \end{itemize}
      Some examples are given below.\nobreak
      \VerbatimInput[firstline=7,lastline=12]{ex01.m}
      See \Figref{ex01} for the output.
      \begin{figure}[ht]
        \centering
        \subfloat[graphics/ex01-1]{\fbox{\psfragfig{graphics/ex01-1}}}\quad
        \subfloat[graphics/ex01-2]{\fbox{\psfragfig{graphics/ex01-2}}}
        \caption{\matlabfrag\ options example showing the {\ttfamily epspad} option at work.}
        \figlabel{ex01}
      \end{figure}
      
      If you wish to show something different in \matlab\ and in the \tex\ document, then you
      can add it into the {\ttfamily UserData} property of the text, with a {\ttfamily matlabfrag:} prefix. This is
      especially useful if you have macros in LaTeX which you want to use, or if you want to use some
      commands not included in the \matlab\ LaTeX installation. In this example, the \verb|\LaTeX| macro
      typesets LaTeX as \LaTeX.
      \VerbatimInput[firstline=7,lastline=11]{ex02.m}
      See \Figref{ex02} for the output.
      \begin{figure}[ht]
        \centering
        \psfragfig{graphics/ex02}
        \caption{UserData example; X-label should say `Plays nice with \LaTeX'.}
        \figlabel{ex02}
      \end{figure}
      
    \subsection{Within LaTeX}\seclabel{in-latex}%
      The \tex\ and \eps\ files can be included in a LaTeX document (pdfLaTeX is treated seperately below)
      by either loading the \psfragname\ package and including the \tex\ file before the \eps\ file:
      \begin{verbatim}
\documentclass{article}
\usepackage{graphicx,psfrag}
\begin{document}
  \include{FileName.tex}
  \includegraphics{FileName.eps}
\end{document}
      \end{verbatim}
      or my preferred method; loading the \pstoolname\ package (v1.2 onwards), and using the
      \verb|\psfragfig| macro:
      \begin{verbatim}
\documentclass{article}
\usepackage{pstool}
\begin{document}
  \psfragfig{FileName}
\end{document}
      \end{verbatim}
      Notice the lack of a file extension in the \verb|\psfragfig| macro. This is a requirement of \pstoolname.
      \pstoolname\ also loads the {\ttfamily graphicx}, {\ttfamily psfrag} and {\ttfamily color} packages (as well as a few others),
      so it is not necessary to explicitly load these packages.
      
      For more information about \pstoolname\ or \psfragname\ please see their respective manuals.
      
      The text replacement performed by \psfragname\ occurs somewhere between the conversion from
      \textsc{dvi} to \textsc{ps} in some seemingly magical process that I don't understand. This means
      that if you look at the \textsc{dvi} output directly you will see a {\ttfamily PSFrag replacements}
      table next to the figure. This is normal, so do not panic.
      
    \subsection{Within pdfLaTeX}%
      There are several ways to include files with PostScript commands (such as \psfragname) in pdfLaTeX such
      as {\ttfamily pst-pdf}, {\ttfamily auto-pst-pdf}, {\ttfamily ps4pdf} and \pstoolname, by far the most useful of which is
      \pstoolname.  The document is set up the same way as the LaTeX example above:
      \begin{verbatim}
\documentclass{article}
\usepackage{pstool}
\begin{document}
  \psfragfig{FileName}
\end{document}
      \end{verbatim}
      The only difference being that \pstoolname\ runs some external processes to run the postscript commands,
      and as such it needs the \verb|-shell-escape| command when being called:\par
{\verb|pdflatex -shell-escape example.tex|}\par\noindent
      where {\ttfamily example.tex} is the code above.
      
    \subsection{Within LyX}\seclabel{in-lyx}%
      Using \matlabfrag\ withing LyX is relatively simple. The steps to setting LyX up with \pstoolname\
      are:
      \begin{enumerate}
        \item Ensure \pstoolname\ is installed in your LaTeX distribution. If you wish to use \pstoolname\
          with non-\textsc{pdf} output, please ensure \pstoolname\ is v1.2 or greater.
        \item Load \pstoolname\ in the document preamble by going to: {\ttfamily Document - Settings}, then
          choose {\ttfamily LaTeX Preamble} in the navigation panel in the left. Insert the line
          \verb|\usepackage{pstool}| and apply the settings.
        \item Enable \verb|-shell-escape| in pdfLaTeX by going to: {\ttfamily Tools - Preferences}, then
          choose {\ttfamily Converters} in the navigation panel in the left. Under {\ttfamily Converter Definitions}
          find the {\ttfamily LaTeX (pdflatex) -> PDF (pdflatex)} entry. Below this change the line\par
          {\verb|pdflatex $$i|}\par\noindent
          to\par
          {\verb|pdflatex -shell-escape $$i|}\par\noindent
          then click {\ttfamily Modify} to the
          upper right, then {\ttfamily Apply} and {\ttfamily Save} down the bottom.
      \end{enumerate}
      Please note that these steps were written for LyX version 1.5.6, but they should be similar for other
      versions.
      
      Once \pstoolname\ is set up for use in LyX, \matlabfrag\ images can be inserted by clicking on the
      \TeX\ button (Insert Tex Code), and placing the {\ttfamily psfragfig} command in:\par
      {\verb|\psfragfig{FileName}|}\par\noindent
      The {\ttfamily ERT} box created with the {\ttfamily psfragfig} command inside it should be placed as you
      would place a normal figure, for example inside a {\ttfamily float:Figure}.
      
      Please also read the discussion at the end of \Secref{in-latex} regarding the \textsc{dvi} output.
      Naturally the same applies for the \textsc{dvi} output from LyX as well.
            
  \section{Creating figures in \matlab}%
    I am now going to suggest a few ways to manipulate figures in \matlab\ before exporting them using \matlabfrag.
    
    \subsection{Sizing}%
      I suggest that the first thing you do when creating a figure is setting it to the size that you want
      in the final document. This way the figure does not need to be resized at any stage, which prevents
      line sizes changing, text over-running borders, etc.  This can be done within \matlab\ using:
      \VerbatimInput[firstline=4,lastline=10]{ex03.m}
      See \Figref{ex03} for the output.
      \begin{figure}[ht]
        \centering
        \psfragfig{graphics/ex03}
        \caption{Resizing example; the figure is sized to \SI{8}{cm} by \SI{3}{cm}.}
        \figlabel{ex03}
      \end{figure}
      
    \subsection{{\ttfamily Color}}%
      The {\ttfamily color} property is a three element vector representing the RGB colour of the text.
      \matlabfrag\ will respect the colour set in \matlab, but may require that you load the {\ttfamily color}
      package in LaTeX (\pstoolname\ implicitly loads the {\ttfamily color} package).
      \VerbatimInput[firstline=7,lastline=11]{ex04.m}
      See \Figref{ex04} for the output.
      \begin{figure}[ht]
        \centering
        \psfragfig{graphics/ex04}
        \caption{{\ttfamily Color} example; the text should be coloured.}
        \figlabel{ex04}
      \end{figure}
    
    \subsection{{\ttfamily FontSize}}%
      All text in \matlab\ figures has a {\ttfamily FontSize} property which \matlabfrag\ respects; the size
      specified in the figure is the size it will be in the LaTeX output document.
      \VerbatimInput[firstline=7,lastline=11]{ex05.m}
      See \Figref{ex05} for the output.
      \begin{figure}[ht]
        \centering
        \psfragfig{graphics/ex05}
        \caption{{\ttfamily FontSize} example; the text should be different sizes.}
        \figlabel{ex05}
      \end{figure}

      
    \subsection{{\ttfamily FontAngle}}%
      {\ttfamily FontAngle} is a property shared by all text in a \matlab\ figure. \matlabfrag\ respects
      {\ttfamily FontAngle} in that {\ttfamily italic} and {\ttfamily oblique} will be italic in the LaTeX output
      document.
      \VerbatimInput[firstline=7,lastline=11]{ex06.m}
      See \Figref{ex06} for the output.
      \begin{figure}[ht]
        \centering
        \psfragfig{graphics/ex06}
        \caption{{\ttfamily FontAngle} example; Oblique and Italic font should both be italic here.}
        \figlabel{ex06}
      \end{figure}
      
    \subsection{{\ttfamily FontWeight}}%
      Another text property from \matlab\ figures that \matlabfrag\ respects is {\ttfamily FontWeight}, such
      that any text set to {\ttfamily bold} will be made bold in the output LaTeX document. {\ttfamily light},
      {\ttfamily demi} and {\ttfamily normal} have no effect on the font in the LaTeX output.
      \VerbatimInput[firstline=7,lastline=12]{ex07.m}
      See \Figref{ex07} for the output.
      \begin{figure}[ht]
        \centering
        \psfragfig{graphics/ex07}
        \caption{{\ttfamily FontWeight} example; bold should be bold, demi and light do not really translate to
        LaTeX, so they should default to normal.}
        \figlabel{ex07}
      \end{figure}
            
    \subsection{{\ttfamily FontName}}%
      {\ttfamily FontName} is a property that \matlabfrag\ \emph{does not} respect, with one exception. One
      of the basic ideas behind using \matlabfrag\ and \psfragname\ is to match the font in figures to
      that of the text, so it does not make sense to use this property, except when it has been set to
      {\ttfamily FixedWidth}, in which case the font in the output LaTeX document will be fixed-width (i.e.,
      \verb|\ttfamily|).
      \VerbatimInput[firstline=7,lastline=11]{ex08.m}
      See \Figref{ex08} for the output.
      \begin{figure}[ht]
        \centering
        \psfragfig{graphics/ex08}
        \caption{{\ttfamily FontName} example; fixed-width should be in a typewriter font, while Comic Sans should
        be changed to the font of the document.}
        \figlabel{ex08}
      \end{figure}
      
    \subsection{{\ttfamily Interpreter}}%
      \matlab\ has three (well, two really) text interpreters that is uses to render the text. These
      are {\ttfamily tex}, {\ttfamily latex} and {\ttfamily none}. I generally don't recommend the use of the default
      {\ttfamily tex} interpreter if rendering anything mathematical and using \matlabfrag, as while the {\ttfamily tex}
      interpreter may render a mathematical expression fine, it may not work in LaTeX.
      
      The {\ttfamily latex} interpreter is especially useful for rendering mathematical formula.
      \VerbatimInput[firstline=7,lastline=11]{ex09.m}
      See \Figref{ex09} for the output.
	    \begin{figure}[ht]
	      \centering
		  \psfragfig{graphics/ex09}
		  \caption{{\ttfamily Interpreter} example; both equations should be the same, despite using a different
		    interpreter in \matlab.}
		  \figlabel{ex09}
	    \end{figure}
      
    \subsection{Multi-line text}%
      Multi-line text can be entered into \matlab\ using either a cell or a two-dimensional character
      matrix. If using a character matrix, \matlabfrag\ preserves all of the white spaces, so if you
      do not wish for this to occur, use cells instead (or check out the {\ttfamily cellstr} function).
      
      \matlabfrag\ recreates multi-line text using a {\ttfamily tabular} environment. If you also use the
      {\ttfamily UserData} property, you will need to manually put the {\ttfamily tabular} environment in, as
      \matlabfrag\ will ignore any {\ttfamily UserData} that is not a string.
      \VerbatimInput[firstline=7,lastline=13]{ex10.m}
      See \Figref{ex10} for the output.
      \begin{figure}[ht]
        \centering
        \psfragfig{graphics/ex10}
        \caption{Multi-line example; three different ways to make multi-line text. 2D arrays, cells
          and using {\ttfamily tabular}.}
        \figlabel{ex10}
      \end{figure}
	  
    \subsection{Custom tick labels}%
      Custom tick labels can be added directly as strings for LaTeX to interpret. In this example,
      the \verb|\ytick| macro has been defined in the preamble of this document as:
      \par{\verb|\newcommand\ytick[1]{\ensuremath{\mathcal R(e^{#1 i})}}|}\par\noindent
      The \matlab\ example code is:
      \VerbatimInput[firstline=7,lastline=12]{ex11.m}
      See \Figref{ex11} for the output.
      \begin{figure}[ht]
        \centering
        \psfragfig{graphics/ex11}
        \caption{Custom tick label example.}
        \figlabel{ex11}
      \end{figure}
      
    \subsection{Tick scaling values}\seclabel{axis-scaling}%
      Tick values automatically generated by \matlab\ may have a scaling value placed at the end
      of the axis. In two-dimensional plots these will be handled by \matlabfrag\ fairly well, but
      in some circumstances, such as three-dimensional plots, they will work badly. An example of
      them being automatically placed in the figure is given in \Figref{comparison02} and the file\par
      {\verb|examples/comparison02.m|}\par\noindent
      When \matlabfrag\ fails to insert the scaling properly, I suggest you manually modify the
      axis ticks, and place the scaling in either the axis label, or place it manually as a text
      object. Example code for doing this is:
      \VerbatimInput[firstline=7,lastline=17]{ex14.m}
      See \Figref{ex14} for the output.
      \begin{figure}[ht]
        \centering
        \psfragfig{graphics/ex14}
        \caption{Manually inserted axis scalings.}
        \figlabel{ex14}
      \end{figure}
      You can read more about why this is necessary in \Secref{adv-axis}.
	  
  \section{Frequently (occasionally) asked questions}%
    \subsection{Why does the output have three digit numbers all through it?}
	
	  Either you are looking at the \eps\ output and that is how it is supposed to look,
	  or there was a problem processing the image with \psfragname. Check out the troubleshooting section
	  below.
	
	  \subsection{Why doesn't the \tex\ file have the
	  \texorpdfstring{{\ttfamily includegraphics}}{includegraphics} statement in it like LaPrint does?}
	
	  Firstly, read the Background section above. With all the image files stored in a
	  {\ttfamily graphics} subdirectory of the main document, I would have to manually open the \tex\ file and
	  insert the path into the that line. This was one of the reasons I wrote \matlabfrag\ in the first
	  place.
	
	  I have considered including an option to produce this output for those who would rather use this
	  behaviour than load \pstoolname\ but decided against it due to the directory problems. From v1.2 onwards
	  \pstoolname\ works fine in LaTeX so there really is no need to manually put the \verb|\includegraphics|
	  line in.
	
    \section{Troubleshooting}
      \subsection{I get an error in \matlab!}
        The only errors that should be generated from \matlabfrag\ are from using it with an old version
        of \matlab\ or calling it with invalid parameters. Unfortunately the error messages generated from
        the {\ttfamily inputParser} class are a bit cryptic, but you should be able to work out if it is a
        parameter error. A simple way to check this is to use an example that is known to work.
        
        No other errors should be generated by \matlabfrag. If you encounter one it is most likely a bug.
        Please email the smallest self-contained example that reproduces the bug to me at
        \par\noindent{\ttfamily zebb.prime+matlabfrag@gmail.com}
      
      \subsection{I get a warning in \matlab!}
        Warnings created by \matlabfrag\ are there for a reason. Read through them, as they explain what
        the problem is and what \matlabfrag\ is doing about it.
      
      \subsection{My graphic exports from \matlab\ fine, but does not compile in LaTeX/pdfLaTeX/LyX!}
	    \begin{enumerate}
	      \item Check that all LaTeX statements in your figure are valid. If you are having problems, try a
	        simple example that is known to work.
	      \item If you have manually loaded the {\ttfamily graphicx} package, make sure you don't have a
	        driver (e.g. {\ttfamily pdftex}) specified.
	      \item If using pdfLaTeX and \pstoolname, check:
	      \begin{enumerate}
	        \item \verb|-shell-escape| option is set when calling pdfLaTeX.
	      \end{enumerate}
	      \item If using LaTeX and \pstoolname, check:
	      \begin{enumerate}
	        \item \pstoolname\ is at least at version 1.2. An alternative location to finding \pstoolname\ is
		      \url{http://github.com/wspr/pstool/}.
	      \end{enumerate}
	      \item If using LaTeX or pdfLaTeX with anything else:
	      \begin{enumerate}
	        \item Install \pstoolname !
	      \end{enumerate}
	      \item If using LyX:
  	    \begin{enumerate}
	        \item Check to see if the \pstoolname\ package is loading by going to {\ttfamily Document -- LaTeX log},
	          and trying to find the \pstoolname\ entry. If it is not found then try repeating the instructions
	          in \Secref{in-lyx}. If it is there, check to see if it is issuing a warning that \verb|-shell-escape|
	          is disabled. If so, follow the instructions in \Secref{in-lyx}.
	      \end{enumerate}
	  \end{enumerate}

    \subsection{My graphic exports from \matlab\ and compiles in LaTeX/\texorpdfstring{\\}{}pdfLaTeX/LyX but doesn't look right!}
      \begin{enumerate}
      \item If you use a {\ttfamily includegraphics} option such as {\ttfamily rotate}, there is a bug
	      in \pstoolname\ prior to v1.2b that caused the option to be applied twice.
      \item The text is running over lines or is cut out!
      \begin{enumerate}
        \item This most likely happens if you resize the image in LaTeX using a \verb|[width=x]| argument.
          It is usually better to size the image in \matlab\ in the first place, and do no further resizing.
        \item If you have mathematical equations, use the {\ttfamily latex} interpreter. This will make the
          text object be roughly the right size, which will let \matlab\ place it properly. This is
          especially useful for legends, as shown in \Figref{comparison01-matlabfrag}.
        \item If the text appears to be cropped from the outside of the graphic then try padding the \eps\
          file using the {\ttfamily epspad} option, or try the {\ttfamily pdfcrop} option
          for \pstoolname:\par
          {\verb|\usepackage[crop=pdfcrop]{pstool}|}
      \end{enumerate}
	    \item A complicated graphic appears different on screen than it does in the document.
	      \begin{enumerate}
	        \item This is most likely an issue caused by a renderer. For more information, read the discussion
	          on renderers in \Secref{renderers}.
	      \end{enumerate}
	    \end{enumerate}

    \section{Advanced usage: a discussion on renderers}\seclabel{renderers}%
      \matlab\ has three different renderers that it uses to render a graphic on screen; {\ttfamily Painters},
      {\ttfamily OpenGL}, and {\ttfamily Z-Buffer}. Of these, only the {\ttfamily Painters} renderer can export
      an image in a native text format which is needed for the text substitution process. If one of the
      other renderers is used the text becomes rasterised, and hence normally no text replacement can take place.
      
      As of \matlabfrag\ {\ttfamily v0.6.0}\footnote{There have been a series of bug fixes since
      {\ttfamily v0.6.0}, so I always recommend using the latest version.},
      an {\ttfamily EXPORTFIG} and {\ttfamily epscombine} style hack
      has been included, which manually places text objects inside rasterised \eps\ images. The result of this
      is that text can now be substituted with LaTeX font and code, but also that all the text now appears
      in the foreground (i.e. all text appears in front of the image, regardless of its placement in the
      original image).
      
      A downside to placing a rasterised image inside an \eps\ file is that it is uncompressed, which creates
      a very large \eps\ file. However, during the conversion to {\ttfamily pdf} the image is compressed to a
      reasonable file size.  If you are using \pstoolname, from {\ttfamily v1.3} it defaults to a lossless
      compression technique similar to \textsc{png}. This is the most suitable compression for technical
      graphics, but should you prefer it, lossy compression can be turned on. For more information, refer to
      the \pstoolname\ documentation.
      
      When printing a rasterised image, the resolution is set using the {\ttfamily dpi} option in \matlabfrag.
      The default is 300, which is print quality.
      
      By default \matlabfrag\ will use the {\ttfamily Painters} renderer unless either the figure has had
      the renderer manually set (which causes the {\ttfamily renderermode} property to become {\ttfamily manual}),
      or the {\ttfamily renderer} option is set in \matlabfrag.
      
      An example of choosing the renderer in \matlabfrag, and increasing the image DPI is given in
      \par{\verb|examples/ex16.m|}\par\noindent
      an extract of which is:
      \VerbatimInput[firstline=7,lastline=15]{ex16.m}
      The output for this example is given in \Figref{ex16}.
      \begin{figure}[ht]
        \centering
        \psfragfig{graphics/ex16}
        \caption{Example showing the use of the {\ttfamily OpenGL} renderer.}
        \figlabel{ex16}
      \end{figure}
      
      An example showing \matlabfrag\ detecting a manually set renderer is given in
      \par{\verb|examples/ex17.m|}\par\noindent
      which is largely based off of the ``Moving the Camera Through a Scene'' example in the \matlab\
      documentation. An extract of this example is:
      \VerbatimInput[firstline=40,lastline=44]{ex17.m}
      The output from this example is given in \Figref{ex17}.
      \begin{figure}[ht]
        \centering
        \psfragfig{graphics/ex17}
        \caption{Example output when the renderer is set to {\ttfamily zbuffer} and shading set to
        {\ttfamily phong}.}
        \figlabel{ex17}
      \end{figure}
      
   \section{Advanced usage: a discussion on axis ticks}\seclabel{adv-axis}%
     Easily the hardest part about writing \matlabfrag\ has been dealing with axis ticks. Their positioning
     and circumstances when the labels are modified are completely hidden from the user. For example, if the
     axis is logarithmic, and the ticks occur at powers of 10, then the {\ttfamily xticklabel} properties
     are written as the power of ten (e.g. 2) and rendered as the 10 to the power of the number (e.g. $10^2$).
     \matlabfrag\ handles this case by detecting whether {\ttfamily xscale} is set to {\ttfamily log}, and
     {\ttfamily xticklabelmode} is set to {\ttfamily auto}. If so, it automatically inserts the appropriate
     LaTeX code. A simple example of this behaviour can be shown with the code:
     \VerbatimInput[firstline=7,lastline=8]{ex15.m}
     which produces \Figref{ex15}.
     \begin{figure}[ht]
       \centering
       \psfragfig{graphics/ex15}
       \caption{An example {\ttfamily bodemag} plot demonstrating ticks of the format $10^x$. Note that
         the axis tick labels are grey. This is due to the {\ttfamily bodemag} function setting them grey
         using the {\ttfamily xcolor} and {\ttfamily ycolor} axis properties.}
       \figlabel{ex15}
     \end{figure}
     
     The above case is fairly simple in comparison to how \matlab\ handles axes that have been scaled by
     a power of ten. For example, if $x\in[0\quad10^{-3}]$, then the x ticks would read, say 
     $[0\quad0.5\quad1]$, and a $\times10^{-3}$ would be placed to the lower right of the x axis.
     Unfortunately, this $\times10^{-3}$ text object is completely hidden from the user, and as soon as
     the tick labels are modified by the user it disappears. Both \matlabfrag\ and {\ttfamily LaPrint}
     have to modify the tick labels to make it possible for \psfragname\ to do text substitution, thus
     we have to manually recreate this scaling. In \matlabfrag\ this is performed by detecting when:
     \begin{enumerate}
       \item {\ttfamily xtickmode} is set to {\ttfamily auto},
       \item {\ttfamily xticklabelmode} is set to {\ttfamily auto}, and
       \item there is a scaling value between the {\ttfamily xtick} and {\ttfamily xticklabel} that is
         equal to a power of ten.
     \end{enumerate}
     In these circumstances, \matlabfrag\ does its best to place the scaling value in the most appropriate
     location. For two-dimensional graphs this isn't too bad, but for three-dimensional plots it is very
     difficult.
     
     The locations for all of the scaling values for all permutations of x and y axis locations
     in a two dimensional plot are shown in \Figref{ex12}. The code for this example is given in:\par
     {\verb|examples/ex12.m|}.
     \begin{figure}[ht]
       \centering
       \psfragfig{graphics/ex12}
       \caption{Axis scale positions for all permutations of the x and y axis locations.}
       \figlabel{ex12}
     \end{figure}
     
     The inbuilt \matlab\ function {\ttfamily plotyy} can create figures with scaling values while
     manually putting in y tick labels (hence {\ttfamily xticklabelmode} is {\ttfamily auto}).
     Unfortunately I don't understand how this works (if someone cares to elighten me, please do), but
     by resetting the {\ttfamily xticklabelmode} to {\ttfamily auto} \matlabfrag\ is able to
     automatically recreate the scaling values. An example of how to do this is given in\par
     {\verb|example/ex13.m|}\par\noindent
     and an example output is given in \Figref{ex13}.
     \begin{figure}[ht]
       \centering
       \psfragfig{graphics/ex13}
       \caption{Output from the {\ttfamily plotyy} function, slightly modified to make \matlabfrag\
         automatically generate the y axis scaling values.}
       \figlabel{ex13}
     \end{figure}
     
     Lastly, support for tick scaling in three-dimensional plots has been added to \matlabfrag\ for
     the default orientation of {\ttfamily surf} plots, such as that shown previously in
     \Figref{comparison02}. If you start to rotate the orientation of the plot it will probably break
     horribly. In these circumstances I suggest you manually place the scaling in the axis label, or
     as a text object. Please read through \Secref{axis-scaling} for some information and an example
     of how to do this.
     
   \section{Additional tests}\seclabel{testing}%
     The examples above don't just serve as examples for new users, they also form part of my testing
     suite for \matlabfrag. Below are some additional tests which are mainly for my development
     purposes.
     \begin{figure}[ht]
       \centering
       \psfragfig{graphics/test01}
       \caption{Output from {\ttfamily test01.m}; making sure multiple legend lines are printed properly.}
       \figlabel{test01}
     \end{figure}
     \begin{figure}[ht]
       \centering
       \psfragfig{graphics/test02}
       \caption{Output from {\ttfamily test02.m}; testing italic font for the main axes property. The
       tick labels aren't italic because they've been rendered in math mode.}
       \figlabel{test02}
     \end{figure}
     \begin{figure}[ht]
       \centering
       \psfragfig{graphics/test03}
       \caption{Output from {\ttfamily test03.m}; bold and fixed-width test with multiple legend entries.}
       \figlabel{test03}
     \end{figure}
     \begin{figure}[ht]
       \centering
       \psfragfig{graphics/test04}
       \caption{Output from {\ttfamily test04.m}; tick labels entered as cells.}
       \figlabel{test04}
     \end{figure}
     \begin{figure}[ht]
       \centering
       \psfragfig{graphics/test05}
       \caption{Output from {\ttfamily test05.m}; tick label alignment test. Y tick labels should be
       nicely right aligned, and x tick labels should be aligned by the numbers excluding any negative
       sign, if present.}
       \figlabel{test05}
     \end{figure}
     \begin{figure}[ht]
       \centering
       \psfragfig{graphics/test06}
       \caption{Output from {\ttfamily test06.m}; legend movement test. The legend should appear slightly
         left and down of centre, and be shrunk to not fill its axes.}
       \figlabel{test06}
     \end{figure}
     \begin{figure}[ht]
       \centering
       \subfloat{\psfragfig{graphics/test07-painters}}
       \quad
       \subfloat{\psfragfig{graphics/test07-opengl}}
       \caption{Output from {\ttfamily test07.m}; different fonts test. The left image is rendered using
       painters, while the right is rendered using opengl.}
       \figlabel{test07}
     \end{figure}
     \begin{figure}[ht]
       \centering
       \subfloat{\psfragfig{graphics/test08-painters}}
       \quad
       \subfloat{\psfragfig{graphics/test08-opengl}}
       \caption{Output from {\ttfamily test08.m}; No text test. The left image is rendered using
       painters, while the right is rendered using opengl.}
       \figlabel{test08}
     \end{figure}
     \begin{figure}
       \centering
       \psfragfig{graphics/test09}
       \caption{Output from {\ttfamily test09.m}; some empty tick labels test. Some of the tick labels
       in both the $x$ and $y$ axes should be empty.}
       \figlabel{test09}
     \end{figure}
\end{document}
